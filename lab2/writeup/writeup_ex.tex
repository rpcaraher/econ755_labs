%%%%%%%%%%%%%%%%%%%%%%%%%%%%%%%%%%%%%%%%%%%%%%%%%%%%%%%%%%%%%%%%%%%%%%%%%%%%%%%%%%%%%%%
%%%%% Preamble
% This is where you load the packages, write custom commands, and design your title page
\documentclass{article} % Specifies the type of document
\usepackage{graphicx} % Required for inserting images
\usepackage{hyperref} % Useful for linking
\usepackage{booktabs} % Helps compile nice-looking table
% \usepackage[authordate,natbib,isbn=false,backend=biber]{biblatex-chicago} % Chicago references

%%%%% Set figure and table WDs

\graphicspath{{/Users/rcaraher/Library/CloudStorage/OneDrive-UniversityofMassachusetts/Academic/Teaching/ECON 755/Problem Sets/Results}}
\newcommand*\InputTable[1]{\input{/Users/rcaraher/Library/CloudStorage/OneDrive-UniversityofMassachusetts/Academic/Teaching/ECON 755/Problem Sets/Results/#1.tex}}

%%%%% Bibliography file
% \addbibresource{references.bib}
 
\title{Econ 755 Lab 2 Writeup}
\author{Ray Caraher}
\date{February 2025}

\begin{document}
\maketitle

%%%%%%%%%%%%%%%%%%%%%%%%%%%%%%%%%%%%%%%%%%%%%%%%%%%%%%%%%%%%%%%%%%%%%%%%%%%%%%%%%%%%%%%
% I use double-comment lines to demark where the body of my document begins
% I use single-comment lines to space sections
%%%%%%%%%%%%%%%%%%%%%%%%%%%%%%%%%%%%%%%%%%%%%%%%%%%%%%%%%%%%%%%%%%%%%%%%%%%%%%%%%%%%%%%

%%%%%%%%%%%%%%%%%%%%%%%%%%%%%%%%%%%%%%%%%%%%%%%%%%%%%%%%%%%%%%%%%%%%%%%%%%%%%%%%%%%%%%%
%%%%% Intrduction

\section{Introduction}
\label{sec:intro} % We need a label for sections and figures to cross-reference them

This is a test example for the Econ 755 problem sets at UMass Amherst,
based on what we covered in lab 2!


% When writing in LaTex,
% it is good to use good coding conventions.
% If you look at the source code for this file,
% you will note that I try to write no more than one clause per line.
% This helps keep my LaTex code legible.
% I also will follow other conventions about lists, figures, links, etc.,
% which would be good for you to take a look at.
% Usually as long as you are consistent then you are fine.

% I also recommend that you use comments frequently in LaTex (lines which start with \%),
% which will not be typeset.
% They are useful for visually organizing your source code,
% making notes to yourself,
% documenting what your code does.

%%%%%%%%%%%%%%%%%%%%%%%%%%%%%%%%%%%%%%%%%%%%%%%%%%%%%%%%%%%%%%%%%%%%%%%%%%%%%%%%%%%%%%%
%%%%% Literature Review


\section{Problem 5}
\label{sec:prob5}

My results for problem 5 are shown in table~\ref{tab:ps1_tabp5_2}.
As can be seen, 
the coefficient estimates are strongly attenuated relative to the bivariate OLS estimate when using matching-based estimators or a rich set of controls.
Estimates of the wage penalty vary between 4\% to 19\%.

Figures~\ref{fig:ps1_dens_1} and~\ref{fig:ps1_dens_2} show the propensity score density plots for the unweighted and weighted control and treatment groups, respectively.
As can be seen, using IPW weights makes the control group look a lot more like the treated group!


%%%%%%%%%%%%%%%%%%%%%%%%%%%%%%%%%%%%%%%%%%%%%%%%%%%%%%%%%%%%%%%%%%%%%%%%%%%%%%%%
%%%%%%%%%%%%%%%%%%%%%%%%%%%%%%%%%%%%%%%%%%%%%%%%%%%%%%%%%%%%%%%%%%%%%%%%%%%%%%%%

% \clearpage
% \newpage
% \printbibliography % Print references section

%%%%%%%%%%%%%%%%%%%%%%%%%%%%%%%%%%%%%%%%%%%%%%%%%%%%%%%%%%%%%%%%%%%%%%%%%%%%%%%%%%%%%%%
%%%%% Figures and Tables

\clearpage
\newpage

\section{Figures and Tables}
\label{sec:figs_tabs}

\begin{figure}[htb]
    \caption{Unweighted Density Plots}
    \label{fig:ps1_dens_1}
    \centering
    \includegraphics[scale=0.50]{ps1_dens_1.pdf}
    \medskip
    \begin{minipage}{0.80 \textwidth}
    {\footnotesize \emph{Notes:} This figure shows the density plots of unweighted propensity scores for the control and
    treated groups, respectively.
    }
    \end{minipage}
\end{figure}

\begin{figure}[htb]
    \caption{Weighted Density Plots}
    \label{fig:ps1_dens_2}
    \centering
    \includegraphics[scale=0.50]{ps1_dens_2.pdf}
    \medskip
    \begin{minipage}{0.80 \textwidth}
    {\footnotesize \emph{Notes:} This figure shows the density plots of IPW weighted propensity scores for the control and
    treated groups, respectively.
    }
    \end{minipage}
\end{figure}

\begin{table}[htb]
    \caption{Regression Results} 
    \label{tab:ps1_tabp5_2}
    \centering
    \InputTable{ps1_tabp5_2}
    \medskip
    \begin{minipage}{0.95 \textwidth}
    {\footnotesize
    \emph{Notes}: This table shows the wage penalty of foreign-born workers using a variety of estimation methods.}
    \end{minipage}
\end{table}


%%%%%%%%%%%%%%%%%%%%%%%%%%%%%%%%%%%%%%%%%%%%%%%%%%%%%%%%%%%%%%%%%%%%%%%%%%%%%%%%%%%%%%%
%%%%%%%%%%%%%%%%%%%%%%%%%%%%%%%%%%%%%%%%%%%%%%%%%%%%%%%%%%%%%%%%%%%%%%%%%%%%%%%%%%%%%%%
\end{document}
