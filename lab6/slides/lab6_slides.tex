% Options for packages loaded elsewhere
\PassOptionsToPackage{unicode}{hyperref}
\PassOptionsToPackage{hyphens}{url}
\documentclass[
  ignorenonframetext,
]{beamer}
\newif\ifbibliography
\usepackage{pgfpages}
\setbeamertemplate{caption}[numbered]
\setbeamertemplate{caption label separator}{: }
\setbeamercolor{caption name}{fg=normal text.fg}
\beamertemplatenavigationsymbolsempty
% remove section numbering
\setbeamertemplate{part page}{
  \centering
  \begin{beamercolorbox}[sep=16pt,center]{part title}
    \usebeamerfont{part title}\insertpart\par
  \end{beamercolorbox}
}
\setbeamertemplate{section page}{
  \centering
  \begin{beamercolorbox}[sep=12pt,center]{section title}
    \usebeamerfont{section title}\insertsection\par
  \end{beamercolorbox}
}
\setbeamertemplate{subsection page}{
  \centering
  \begin{beamercolorbox}[sep=8pt,center]{subsection title}
    \usebeamerfont{subsection title}\insertsubsection\par
  \end{beamercolorbox}
}
% Prevent slide breaks in the middle of a paragraph
\widowpenalties 1 10000
\raggedbottom
\AtBeginPart{
  \frame{\partpage}
}
\AtBeginSection{
  \ifbibliography
  \else
    \frame{\sectionpage}
  \fi
}
\AtBeginSubsection{
  \frame{\subsectionpage}
}
\usepackage{iftex}
\ifPDFTeX
  \usepackage[T1]{fontenc}
  \usepackage[utf8]{inputenc}
  \usepackage{textcomp} % provide euro and other symbols
\else % if luatex or xetex
  \usepackage{unicode-math} % this also loads fontspec
  \defaultfontfeatures{Scale=MatchLowercase}
  \defaultfontfeatures[\rmfamily]{Ligatures=TeX,Scale=1}
\fi
\usepackage{lmodern}
\ifPDFTeX\else
  % xetex/luatex font selection
\fi
% Use upquote if available, for straight quotes in verbatim environments
\IfFileExists{upquote.sty}{\usepackage{upquote}}{}
\IfFileExists{microtype.sty}{% use microtype if available
  \usepackage[]{microtype}
  \UseMicrotypeSet[protrusion]{basicmath} % disable protrusion for tt fonts
}{}
\makeatletter
\@ifundefined{KOMAClassName}{% if non-KOMA class
  \IfFileExists{parskip.sty}{%
    \usepackage{parskip}
  }{% else
    \setlength{\parindent}{0pt}
    \setlength{\parskip}{6pt plus 2pt minus 1pt}}
}{% if KOMA class
  \KOMAoptions{parskip=half}}
\makeatother
\usepackage{color}
\usepackage{fancyvrb}
\newcommand{\VerbBar}{|}
\newcommand{\VERB}{\Verb[commandchars=\\\{\}]}
\DefineVerbatimEnvironment{Highlighting}{Verbatim}{commandchars=\\\{\}}
% Add ',fontsize=\small' for more characters per line
\usepackage{framed}
\definecolor{shadecolor}{RGB}{248,248,248}
\newenvironment{Shaded}{\begin{snugshade}}{\end{snugshade}}
\newcommand{\AlertTok}[1]{\textcolor[rgb]{0.94,0.16,0.16}{#1}}
\newcommand{\AnnotationTok}[1]{\textcolor[rgb]{0.56,0.35,0.01}{\textbf{\textit{#1}}}}
\newcommand{\AttributeTok}[1]{\textcolor[rgb]{0.13,0.29,0.53}{#1}}
\newcommand{\BaseNTok}[1]{\textcolor[rgb]{0.00,0.00,0.81}{#1}}
\newcommand{\BuiltInTok}[1]{#1}
\newcommand{\CharTok}[1]{\textcolor[rgb]{0.31,0.60,0.02}{#1}}
\newcommand{\CommentTok}[1]{\textcolor[rgb]{0.56,0.35,0.01}{\textit{#1}}}
\newcommand{\CommentVarTok}[1]{\textcolor[rgb]{0.56,0.35,0.01}{\textbf{\textit{#1}}}}
\newcommand{\ConstantTok}[1]{\textcolor[rgb]{0.56,0.35,0.01}{#1}}
\newcommand{\ControlFlowTok}[1]{\textcolor[rgb]{0.13,0.29,0.53}{\textbf{#1}}}
\newcommand{\DataTypeTok}[1]{\textcolor[rgb]{0.13,0.29,0.53}{#1}}
\newcommand{\DecValTok}[1]{\textcolor[rgb]{0.00,0.00,0.81}{#1}}
\newcommand{\DocumentationTok}[1]{\textcolor[rgb]{0.56,0.35,0.01}{\textbf{\textit{#1}}}}
\newcommand{\ErrorTok}[1]{\textcolor[rgb]{0.64,0.00,0.00}{\textbf{#1}}}
\newcommand{\ExtensionTok}[1]{#1}
\newcommand{\FloatTok}[1]{\textcolor[rgb]{0.00,0.00,0.81}{#1}}
\newcommand{\FunctionTok}[1]{\textcolor[rgb]{0.13,0.29,0.53}{\textbf{#1}}}
\newcommand{\ImportTok}[1]{#1}
\newcommand{\InformationTok}[1]{\textcolor[rgb]{0.56,0.35,0.01}{\textbf{\textit{#1}}}}
\newcommand{\KeywordTok}[1]{\textcolor[rgb]{0.13,0.29,0.53}{\textbf{#1}}}
\newcommand{\NormalTok}[1]{#1}
\newcommand{\OperatorTok}[1]{\textcolor[rgb]{0.81,0.36,0.00}{\textbf{#1}}}
\newcommand{\OtherTok}[1]{\textcolor[rgb]{0.56,0.35,0.01}{#1}}
\newcommand{\PreprocessorTok}[1]{\textcolor[rgb]{0.56,0.35,0.01}{\textit{#1}}}
\newcommand{\RegionMarkerTok}[1]{#1}
\newcommand{\SpecialCharTok}[1]{\textcolor[rgb]{0.81,0.36,0.00}{\textbf{#1}}}
\newcommand{\SpecialStringTok}[1]{\textcolor[rgb]{0.31,0.60,0.02}{#1}}
\newcommand{\StringTok}[1]{\textcolor[rgb]{0.31,0.60,0.02}{#1}}
\newcommand{\VariableTok}[1]{\textcolor[rgb]{0.00,0.00,0.00}{#1}}
\newcommand{\VerbatimStringTok}[1]{\textcolor[rgb]{0.31,0.60,0.02}{#1}}
\newcommand{\WarningTok}[1]{\textcolor[rgb]{0.56,0.35,0.01}{\textbf{\textit{#1}}}}
\setlength{\emergencystretch}{3em} % prevent overfull lines
\providecommand{\tightlist}{%
  \setlength{\itemsep}{0pt}\setlength{\parskip}{0pt}}
\usepackage{bookmark}
\IfFileExists{xurl.sty}{\usepackage{xurl}}{} % add URL line breaks if available
\urlstyle{same}
\hypersetup{
  pdftitle={Lab 6},
  pdfauthor={Ray Caraher},
  hidelinks,
  pdfcreator={LaTeX via pandoc}}

\title{Lab 6}
\author{Ray Caraher}
\date{2025-05-06}

\begin{document}
\frame{\titlepage}

\section{Setup}\label{setup}

\begin{frame}[fragile]{Setting up our script}
\phantomsection\label{setting-up-our-script}
Before we get into any real coding, let's make sure that the preamble
for our code looks good. Here is how I set it up:

\tiny

\begin{Shaded}
\begin{Highlighting}[]
\DocumentationTok{\#\# Load packages}
\FunctionTok{library}\NormalTok{(haven)}
\FunctionTok{library}\NormalTok{(fixest)}
\FunctionTok{library}\NormalTok{(tidyverse)}
\DocumentationTok{\#\# Set options}

\FunctionTok{options}\NormalTok{(}\AttributeTok{scipen =} \DecValTok{999}\NormalTok{)}

\DocumentationTok{\#\# Clear environment}

\FunctionTok{rm}\NormalTok{(}\AttributeTok{list =} \FunctionTok{ls}\NormalTok{())}

\DocumentationTok{\#\# Set directories}

\NormalTok{base\_directory }\OtherTok{\textless{}{-}} \StringTok{\textquotesingle{}/Users/rcaraher/Library/CloudStorage/OneDrive{-}UniversityofMassachusetts/Academic/Teaching/ECON 755/Problem Sets\textquotesingle{}}
\NormalTok{data\_directory }\OtherTok{\textless{}{-}} \FunctionTok{file.path}\NormalTok{(base\_directory, }\StringTok{\textquotesingle{}Data\textquotesingle{}}\NormalTok{)}
\NormalTok{results\_directory }\OtherTok{\textless{}{-}} \FunctionTok{file.path}\NormalTok{(base\_directory, }\StringTok{\textquotesingle{}Results\textquotesingle{}}\NormalTok{)}
\end{Highlighting}
\end{Shaded}
\end{frame}

\section{Instrumental Variables in R}\label{instrumental-variables-in-r}

\begin{frame}{Overview of IVs}
\phantomsection\label{overview-of-ivs}
Instrumental variables is another route to ``causal'' inference

\begin{itemize}
\tightlist
\item
  In DiD approach, only looking at changes in treatment status that are
  exogenous (caused by policy changes, lottery, natural experiment,
  etc.)
\item
  In IV approach, only looking at variation in outcome that is
  correlated with variation in an exogenous variable (the instrument)
\end{itemize}
\end{frame}

\begin{frame}{Identifying IVs}
\phantomsection\label{identifying-ivs}
We want to estimate the following regression:

\[ Y = \beta_{0} + \beta_{1}X + \epsilon \]

However, we have reason to believe that \(X\) is not exogenous.

Examples:

\begin{enumerate}
\tightlist
\item
  Effect of schooling (\(X\)) on wages (\(Y\)): Some unobservable
  omitted variable (e.g., ability) is correlated with both \(X\) and
  \(Y\)
\item
  Effect of insurance (\(X\)) on health (\(Y\)): Selection in that
  healthier individuals (\(Y\)) may be more likely to get insurance
  (\(X\))
\item
  Effect of policing (\(X\)) on crime (\(Y\)): Reverse causality as
  police (\(X\)) are often deployed to areas with high crime rates
  (\(Y\))
\end{enumerate}
\end{frame}

\begin{frame}{Identifying IVs}
\phantomsection\label{identifying-ivs-1}
In these cases, \(X\) will not have a valid causal interpretation.

What can we do?

One solution is to identify an \textbf{instrument \(Z\)} which is
correlated with \(X\) and correlated with \(Y\) \emph{only through its
correlation with \(X\)}

In other words, the instrument has be \textbf{relevant and excusable}
\end{frame}

\begin{frame}{Conditions for a Valid Instrument}
\phantomsection\label{conditions-for-a-valid-instrument}
If the model is \[Y = \beta_0 + \beta_1 X + \epsilon\]

then a valid instrument \(Z\) must be:

\begin{enumerate}
\tightlist
\item
  Relevant: \(Cov(Z,X) \neq 0\) \(\rightarrow\) \(Z\) must be correlated
  with \(X\)
\item
  Excludable: \(Cov(Z,\epsilon) = 0\) \(\rightarrow\) \(Z\) must
  \emph{not} be correlated with the error term
\end{enumerate}
\end{frame}

\begin{frame}{Estimation via 2SLS: Stage 1}
\phantomsection\label{estimation-via-2sls-stage-1}
To estimate \(Y = \beta X + \epsilon\) with instrument \(Z\), use 2SLS:

Stage 1: Regress \(X\) on \(Z\) (get predicted \(\hat{X}\)):

\[ X = \pi_0 + \pi_{1}Z + \eta \] This is called the \textbf{first
stage}
\end{frame}

\begin{frame}{Notes about the First Stage - Weak Instruments}
\phantomsection\label{notes-about-the-first-stage---weak-instruments}
If \(\pi_1\) is close to zero, \(Z\) is a weak instrument:

\begin{itemize}
\tightlist
\item
  \(\hat{X}\) contains little exogenous variation
\item
  Stop here: 2SLS estimates become biased and unreliable (can be worse
  than OLS)
\end{itemize}

Can use statistical tests to look for weak instruments, most common
being the \textbf{F-statistic}
\end{frame}

\begin{frame}{Estimation via 2SLS: Stage 2}
\phantomsection\label{estimation-via-2sls-stage-2}
Stage 2: Regress \(Y\) on \(\hat{X}\):

\[ Y = \beta_0 + \beta_1 \hat{X} + \mu \]
\end{frame}

\begin{frame}{Why use 2SLS?}
\phantomsection\label{why-use-2sls}
\begin{itemize}
\tightlist
\item
  OLS is biased when \(X\) is endogenous
\item
  IV isolates exogenous variation in \(X\)
\item
  2SLS is consistent (though often less efficient than OLS)
\end{itemize}
\end{frame}

\section{2SLS in R}\label{sls-in-r}

\begin{frame}{Estimating IVs in R}
\phantomsection\label{estimating-ivs-in-r}
\begin{itemize}
\tightlist
\item
  Estimating 2SLS in R is a simple extension from our normal regression
  tools
\end{itemize}
\end{frame}

\begin{frame}{The setting}
\phantomsection\label{the-setting}
What is the effect of fertility on labor supply?

Important empirical question for many reasons:

\begin{itemize}
\tightlist
\item
  Having children could push women out of the labor force for some time,
  having career implications
\item
  Having children may lead to increased premiums for fathers in the
  labor market
\item
  More educated/higher class families may be differentially affected by
  fertility
\end{itemize}
\end{frame}

\begin{frame}{The setting}
\phantomsection\label{the-setting-1}
Want to estimate the following:

\[ Y = \beta_0 + \beta_1 X + \epsilon \] where \(Y\) is a labor market
outcome (hours worked, employment status, wages, etc.) and \(X\) is
fertility (number of children, having any children, etc.)

However, naive estimates of \(X\) on \(Y\) may not be credible if we aim
to estimate a causal effect for many reasons:

\begin{enumerate}
\tightlist
\item
  Families may time when to have children based on labor market factors
  (reverse causality)
\item
  Families may have children when they expect their labor market
  outcomes to improve (OVB)
\item
  Families that are less resource/time constrained may have more
  children (selection)
\end{enumerate}
\end{frame}

\begin{frame}{The setting}
\phantomsection\label{the-setting-2}
Angrist and Evans (1998) propose an instrumental variable to overcome
these biases: \textbf{sex composition of current children}

Relevance:

\begin{itemize}
\tightlist
\item
  Families have a strong desire to have mixed-sex children and will
  increase fertility to do so

  \begin{itemize}
  \tightlist
  \item
    If you have two girls, will likely have a third child in an attempt
    to have a son
  \item
    But if you have one boy and one girl already, less likely to have a
    third child
  \end{itemize}
\end{itemize}

Exculdable:

\begin{itemize}
\tightlist
\item
  Initial sex of children is \emph{randomly assigned}
\end{itemize}

Therefore, having two same-sex children is (arguably) a valid
\textbf{instrument for fertility}
\end{frame}

\begin{frame}[fragile]{Getting data}
\phantomsection\label{getting-data}
Let's read in the data and take a look.

\scriptsize

\begin{Shaded}
\begin{Highlighting}[]
\NormalTok{ae\_pums }\OtherTok{\textless{}{-}} \FunctionTok{read\_dta}\NormalTok{(}\FunctionTok{file.path}\NormalTok{(data\_directory, }\StringTok{"angrist\_evans\_data.dta"}\NormalTok{))}

\FunctionTok{glimpse}\NormalTok{(ae\_pums)}
\end{Highlighting}
\end{Shaded}

\begin{verbatim}
## Rows: 402,014
## Columns: 11
## $ twins_1          <dbl> 0, 0, 0, 0, 0, 0, 0, 0, 0, 0, 0, 0, 0, 0, 0, 0, 0, 0,~
## $ mom_weeks_worked <dbl> 0, 52, 30, 0, 0, 0, 22, 26, 40, 0, 52, 0, 52, 0, 52, ~
## $ kidcount         <dbl> 2, 2, 2, 2, 2, 3, 2, 3, 2, 2, 2, 2, 2, 2, 4, 3, 2, 2,~
## $ mom_worked       <dbl> 0, 1, 1, 0, 0, 0, 1, 1, 1, 0, 1, 0, 1, 0, 1, 0, 1, 1,~
## $ twins_2          <dbl> 0, 0, 0, 0, 0, 0, 0, 0, 0, 0, 0, 0, 0, 0, 0, 0, 0, 0,~
## $ moreths          <dbl> 0, 0, 0, 0, 1, 0, 1, 0, 0, 1, 0, 1, 0, 0, 0, 0, 0, 0,~
## $ whitem           <dbl> 1, 1, 1, 1, 0, 1, 1, 1, 1, 1, 1, 1, 1, 1, 1, 1, 1, 1,~
## $ blackm           <dbl> 0, 0, 0, 0, 1, 0, 0, 0, 0, 0, 0, 0, 0, 0, 0, 0, 0, 0,~
## $ morekids         <dbl> 0, 0, 0, 0, 0, 1, 0, 1, 0, 0, 0, 0, 0, 0, 1, 1, 0, 0,~
## $ hispm            <dbl> 0, 0, 0, 0, 0, 0, 0, 0, 0, 0, 0, 0, 0, 0, 0, 0, 0, 0,~
## $ samesex          <dbl> 0, 0, 0, 0, 0, 1, 1, 1, 0, 0, 1, 0, 0, 0, 1, 1, 0, 1,~
\end{verbatim}
\end{frame}

\begin{frame}[fragile]{Looking at data}
\phantomsection\label{looking-at-data}
Let's read in the data and take a look at some descriptives:

\scriptsize

\begin{Shaded}
\begin{Highlighting}[]
\NormalTok{desc\_tab }\OtherTok{\textless{}{-}}\NormalTok{ ae\_pums }\SpecialCharTok{\%\textgreater{}\%}
  \FunctionTok{summarise}\NormalTok{(}\AttributeTok{mean\_kids =} \FunctionTok{mean}\NormalTok{(kidcount, }\AttributeTok{na.rm =}\NormalTok{ T),}
            \AttributeTok{samesex =} \FunctionTok{mean}\NormalTok{(samesex, }\AttributeTok{na.rm =}\NormalTok{ T))}
\NormalTok{desc\_tab}
\end{Highlighting}
\end{Shaded}

\begin{verbatim}
## # A tibble: 1 x 2
##   mean_kids samesex
##       <dbl>   <dbl>
## 1      2.56   0.505
\end{verbatim}
\end{frame}

\begin{frame}[fragile]{Looking at data}
\phantomsection\label{looking-at-data-1}
Let's read in the data and take a look at some descriptives

\scriptsize

\begin{Shaded}
\begin{Highlighting}[]
\NormalTok{ae\_pums }\SpecialCharTok{\%\textgreater{}\%}
  \FunctionTok{count}\NormalTok{(kidcount)}
\end{Highlighting}
\end{Shaded}

\begin{verbatim}
## # A tibble: 11 x 2
##    kidcount      n
##       <dbl>  <int>
##  1        2 239150
##  2        3 117203
##  3        4  33577
##  4        5   9046
##  5        6   2160
##  6        7    607
##  7        8    205
##  8        9     39
##  9       10     18
## 10       11      7
## 11       12      2
\end{verbatim}
\end{frame}

\begin{frame}[fragile]{Looking at data}
\phantomsection\label{looking-at-data-2}
Let's read in the data and take a look at some descriptives

\scriptsize

\begin{Shaded}
\begin{Highlighting}[]
\NormalTok{ae\_pums }\SpecialCharTok{\%\textgreater{}\%}
  \FunctionTok{group\_by}\NormalTok{(samesex) }\SpecialCharTok{\%\textgreater{}\%}
  \FunctionTok{summarise}\NormalTok{(}\AttributeTok{morekids =} \FunctionTok{mean}\NormalTok{(morekids, }\AttributeTok{na.rm =}\NormalTok{ T))}
\end{Highlighting}
\end{Shaded}

\begin{verbatim}
## # A tibble: 2 x 2
##   samesex morekids
##     <dbl>    <dbl>
## 1       0    0.375
## 2       1    0.435
\end{verbatim}
\end{frame}

\begin{frame}[fragile]{Estimating OLS}
\phantomsection\label{estimating-ols}
Let's first estimate using OLS the effect of having at least three kids
on the likelihood mom worked:

\tiny

\begin{Shaded}
\begin{Highlighting}[]
\NormalTok{ae\_pums }\OtherTok{\textless{}{-}}\NormalTok{ ae\_pums }\SpecialCharTok{\%\textgreater{}\%}
  \FunctionTok{mutate}\NormalTok{(}\AttributeTok{mt2kids =} \FunctionTok{case\_when}\NormalTok{(kidcount }\SpecialCharTok{\textgreater{}} \DecValTok{2} \SpecialCharTok{\textasciitilde{}} \DecValTok{1}\NormalTok{,}
                             \FunctionTok{is.na}\NormalTok{(kidcount) }\SpecialCharTok{\textasciitilde{}} \ConstantTok{NA\_real\_}\NormalTok{,}
                             \ConstantTok{TRUE} \SpecialCharTok{\textasciitilde{}} \DecValTok{0}\NormalTok{))}

\NormalTok{mworked\_ols }\OtherTok{\textless{}{-}} \FunctionTok{feols}\NormalTok{(mom\_worked }\SpecialCharTok{\textasciitilde{}}\NormalTok{ mt2kids }\SpecialCharTok{+}\NormalTok{ whitem }\SpecialCharTok{+}\NormalTok{ blackm }\SpecialCharTok{+}\NormalTok{ hispm }\SpecialCharTok{+}\NormalTok{ moreths, }
                     \AttributeTok{data =}\NormalTok{ ae\_pums)}
\FunctionTok{summary}\NormalTok{(mworked\_ols)}
\end{Highlighting}
\end{Shaded}

\begin{verbatim}
## OLS estimation, Dep. Var.: mom_worked
## Observations: 402,014
## Standard-errors: IID 
##              Estimate Std. Error   t value              Pr(>|t|)    
## (Intercept)  0.599837   0.004663 128.63257             < 2.2e-16 ***
## mt2kids     -0.121478   0.001587 -76.52607             < 2.2e-16 ***
## whitem      -0.017575   0.004642  -3.78571 0.0001532915018691329 ***
## blackm       0.107326   0.005067  21.17960             < 2.2e-16 ***
## hispm       -0.049719   0.006368  -7.80765 0.0000000000000058403 ***
## moreths      0.059676   0.001712  34.86374             < 2.2e-16 ***
## ---
## Signif. codes:  0 '***' 0.001 '**' 0.01 '*' 0.05 '.' 0.1 ' ' 1
## RMSE: 0.48968   Adj. R2: 0.024116
\end{verbatim}

This is pretty close to the estimated effect in AE (1998) of
\texttt{-0.176} (see table 5, row 1 in the NBER working paper draft)
\end{frame}

\begin{frame}[fragile]{Estimating OLS}
\phantomsection\label{estimating-ols-1}
Let's now estimate using OLS the effect of having at least three kids on
the number of weeks mom worked:

\tiny

\begin{Shaded}
\begin{Highlighting}[]
\NormalTok{mweeks\_ols }\OtherTok{\textless{}{-}} \FunctionTok{feols}\NormalTok{(mom\_weeks\_worked }\SpecialCharTok{\textasciitilde{}}\NormalTok{ mt2kids }\SpecialCharTok{+}\NormalTok{ whitem }\SpecialCharTok{+}\NormalTok{ blackm }\SpecialCharTok{+}\NormalTok{ hispm }\SpecialCharTok{+}\NormalTok{ moreths, }
                     \AttributeTok{data =}\NormalTok{ ae\_pums)}
\FunctionTok{summary}\NormalTok{(mweeks\_ols)}
\end{Highlighting}
\end{Shaded}

\begin{verbatim}
## OLS estimation, Dep. Var.: mom_weeks_worked
## Observations: 402,014
## Standard-errors: IID 
##             Estimate Std. Error   t value               Pr(>|t|)    
## (Intercept) 23.44623   0.208961 112.20389              < 2.2e-16 ***
## mt2kids     -6.11931   0.071133 -86.02610              < 2.2e-16 ***
## whitem      -1.70262   0.208033  -8.18435 0.00000000000000027457 ***
## blackm       5.26005   0.227075  23.16443              < 2.2e-16 ***
## hispm       -3.01258   0.285356 -10.55726              < 2.2e-16 ***
## moreths      2.35799   0.076702  30.74229              < 2.2e-16 ***
## ---
## Signif. codes:  0 '***' 0.001 '**' 0.01 '*' 0.05 '.' 0.1 ' ' 1
## RMSE: 21.9   Adj. R2: 0.030178
\end{verbatim}
\end{frame}

\begin{frame}[fragile]{Estimating 2SLS}
\phantomsection\label{estimating-2sls}
We have already discussed why a naive regression of fertility on labor
market outcomes may be endogenous. Now, let's instrument using \(Z\) as
an indicator for if the family had 2 kids of the same sex at birth.

We will first implement it by-hand, then using \texttt{feols()}
\end{frame}

\begin{frame}[fragile]{Estimating the First Stage}
\phantomsection\label{estimating-the-first-stage}
Let's estimate the first stage.

When we use covariates, its important to include them on the RHS of the
first-stage regression as well!

\tiny

\begin{Shaded}
\begin{Highlighting}[]
\NormalTok{fs }\OtherTok{\textless{}{-}} \FunctionTok{feols}\NormalTok{(mt2kids }\SpecialCharTok{\textasciitilde{}}\NormalTok{ samesex }\SpecialCharTok{+}\NormalTok{ whitem }\SpecialCharTok{+}\NormalTok{ blackm }\SpecialCharTok{+}\NormalTok{ hispm }\SpecialCharTok{+}\NormalTok{ moreths,}
            \AttributeTok{data =}\NormalTok{ ae\_pums)}
\FunctionTok{summary}\NormalTok{(fs)}
\end{Highlighting}
\end{Shaded}

\begin{verbatim}
## OLS estimation, Dep. Var.: mt2kids
## Observations: 402,014
## Standard-errors: IID 
##              Estimate Std. Error  t value  Pr(>|t|)    
## (Intercept)  0.435078   0.004630  93.9641 < 2.2e-16 ***
## samesex      0.059407   0.001532  38.7794 < 2.2e-16 ***
## whitem      -0.052779   0.004603 -11.4656 < 2.2e-16 ***
## blackm       0.067258   0.005024  13.3866 < 2.2e-16 ***
## hispm        0.096116   0.006313  15.2241 < 2.2e-16 ***
## moreths     -0.096315   0.001691 -56.9677 < 2.2e-16 ***
## ---
## Signif. codes:  0 '***' 0.001 '**' 0.01 '*' 0.05 '.' 0.1 ' ' 1
## RMSE: 0.485619   Adj. R2: 0.021451
\end{verbatim}
\end{frame}

\begin{frame}[fragile]{Estimating the First Stage}
\phantomsection\label{estimating-the-first-stage-1}
The regression is significant and positive, suggesting that having 2
kids of the same sex significant increases your probability of having a
third.

Now let's generate the predicted value of \(X\) using the variation in
\texttt{samesex}

\scriptsize

\begin{Shaded}
\begin{Highlighting}[]
\NormalTok{ae\_pums }\OtherTok{\textless{}{-}}\NormalTok{ ae\_pums }\SpecialCharTok{\%\textgreater{}\%}
  \FunctionTok{mutate}\NormalTok{(}\AttributeTok{X\_hat =} \FunctionTok{predict}\NormalTok{(fs, }\AttributeTok{type =} \StringTok{"response"}\NormalTok{))}
\end{Highlighting}
\end{Shaded}
\end{frame}

\begin{frame}[fragile]{Estimating the second stage}
\phantomsection\label{estimating-the-second-stage}
Now, we use the predicted values from the first stage to estimate the
effect of the \textbf{exogenous} part of fertility on labor market
outcomes:

\tiny

\begin{Shaded}
\begin{Highlighting}[]
\NormalTok{mworked\_s2 }\OtherTok{\textless{}{-}} \FunctionTok{feols}\NormalTok{(mom\_worked }\SpecialCharTok{\textasciitilde{}}\NormalTok{ X\_hat }\SpecialCharTok{+}\NormalTok{ whitem }\SpecialCharTok{+}\NormalTok{ blackm }\SpecialCharTok{+}\NormalTok{ hispm }\SpecialCharTok{+}\NormalTok{ moreths, }
                     \AttributeTok{data =}\NormalTok{ ae\_pums)}
\FunctionTok{summary}\NormalTok{(mworked\_s2)}
\end{Highlighting}
\end{Shaded}

\begin{verbatim}
## OLS estimation, Dep. Var.: mom_worked
## Observations: 402,014
## Standard-errors: IID 
##              Estimate Std. Error  t value           Pr(>|t|)    
## (Intercept)  0.603134   0.013028 46.29646          < 2.2e-16 ***
## X_hat       -0.128571   0.026190 -4.90908 0.0000009154029933 ***
## whitem      -0.017947   0.004873 -3.68286 0.0002306660443972 ***
## blackm       0.107805   0.005401 19.95974          < 2.2e-16 ***
## hispm       -0.049035   0.006892 -7.11519 0.0000000000011195 ***
## moreths      0.058992   0.003052 19.32714          < 2.2e-16 ***
## ---
## Signif. codes:  0 '***' 0.001 '**' 0.01 '*' 0.05 '.' 0.1 ' ' 1
## RMSE: 0.493219   Adj. R2: 0.00996
\end{verbatim}

\tiny

\begin{Shaded}
\begin{Highlighting}[]
\NormalTok{mweeks\_s2 }\OtherTok{\textless{}{-}} \FunctionTok{feols}\NormalTok{(mom\_weeks\_worked }\SpecialCharTok{\textasciitilde{}}\NormalTok{ X\_hat }\SpecialCharTok{+}\NormalTok{ whitem }\SpecialCharTok{+}\NormalTok{ blackm }\SpecialCharTok{+}\NormalTok{ hispm }\SpecialCharTok{+}\NormalTok{ moreths, }
                     \AttributeTok{data =}\NormalTok{ ae\_pums)}
\FunctionTok{summary}\NormalTok{(mweeks\_s2)}
\end{Highlighting}
\end{Shaded}

\begin{verbatim}
## OLS estimation, Dep. Var.: mom_weeks_worked
## Observations: 402,014
## Standard-errors: IID 
##             Estimate Std. Error  t value              Pr(>|t|)    
## (Intercept) 23.45461   0.584881 40.10148             < 2.2e-16 ***
## X_hat       -6.13733   1.175832 -5.21956 0.0000001794336473852 ***
## whitem      -1.70356   0.218786 -7.78644 0.0000000000000069086 ***
## blackm       5.26127   0.242485 21.69727             < 2.2e-16 ***
## hispm       -3.01084   0.309403 -9.73114             < 2.2e-16 ***
## moreths      2.35626   0.137034 17.19464             < 2.2e-16 ***
## ---
## Signif. codes:  0 '***' 0.001 '**' 0.01 '*' 0.05 '.' 0.1 ' ' 1
## RMSE: 22.1   Adj. R2: 0.012392
\end{verbatim}

These results are also close to the AE (1998) results in table 5 of
\texttt{-0.121} for any employment and \texttt{-5.68} for weeks worked.
\end{frame}

\begin{frame}[fragile]{Estimating 2SLS using a built-in routine}
\phantomsection\label{estimating-2sls-using-a-built-in-routine}
The \texttt{feols()} function we have been working with has built-in
ability to estimate 2SLS.

Simply add another \texttt{\textbar{}} after the fixed-effects using
\texttt{X\ \textasciitilde{}\ Z} formula syntax.

The control variables on the immediate RHS of the formula will
automatically be included.

Since we have no fixed-effects here, we can just use a \texttt{0}.

Our results perfectly match those computations we did by-hand.
\end{frame}

\begin{frame}[fragile]{Estimating 2SLS using a built-in routine}
\phantomsection\label{estimating-2sls-using-a-built-in-routine-1}
\tiny

\begin{Shaded}
\begin{Highlighting}[]
\NormalTok{mworked\_tsls }\OtherTok{\textless{}{-}} \FunctionTok{feols}\NormalTok{(mom\_worked }\SpecialCharTok{\textasciitilde{}}\NormalTok{ whitem }\SpecialCharTok{+}\NormalTok{ blackm }\SpecialCharTok{+}\NormalTok{ hispm }\SpecialCharTok{+}\NormalTok{ moreths }\SpecialCharTok{|} 
                \DecValTok{0} \SpecialCharTok{|} 
\NormalTok{                mt2kids }\SpecialCharTok{\textasciitilde{}}\NormalTok{ samesex,}
              \AttributeTok{data =}\NormalTok{ ae\_pums)}
\FunctionTok{summary}\NormalTok{(mworked\_tsls)}
\end{Highlighting}
\end{Shaded}

\begin{verbatim}
## TSLS estimation - Dep. Var.: mom_worked
##                   Endo.    : mt2kids
##                   Instr.   : samesex
## Second stage: Dep. Var.: mom_worked
## Observations: 402,014
## Standard-errors: IID 
##              Estimate Std. Error  t value            Pr(>|t|)    
## (Intercept)  0.603134   0.012934 46.62990           < 2.2e-16 ***
## fit_mt2kids -0.128571   0.026003 -4.94444 0.00000076395140220 ***
## whitem      -0.017947   0.004838 -3.70938 0.00020779515550137 ***
## blackm       0.107805   0.005362 20.10349           < 2.2e-16 ***
## hispm       -0.049035   0.006842 -7.16643 0.00000000000077108 ***
## moreths      0.058992   0.003030 19.46634           < 2.2e-16 ***
## ---
## Signif. codes:  0 '***' 0.001 '**' 0.01 '*' 0.05 '.' 0.1 ' ' 1
## RMSE: 0.489692   Adj. R2: 0.024068
## F-test (1st stage), mt2kids: stat = 1,503.8     , p < 2.2e-16 , on 1 and 402,008 DoF.
##                  Wu-Hausman: stat =     0.074685, p = 0.784633, on 1 and 402,007 DoF.
\end{verbatim}
\end{frame}

\begin{frame}[fragile]{Estimating 2SLS using a built-in routine}
\phantomsection\label{estimating-2sls-using-a-built-in-routine-2}
\tiny

\begin{Shaded}
\begin{Highlighting}[]
\NormalTok{mweeks\_tsls }\OtherTok{\textless{}{-}} \FunctionTok{feols}\NormalTok{(mom\_weeks\_worked }\SpecialCharTok{\textasciitilde{}}\NormalTok{ whitem }\SpecialCharTok{+}\NormalTok{ blackm }\SpecialCharTok{+}\NormalTok{ hispm }\SpecialCharTok{+}\NormalTok{ moreths }\SpecialCharTok{|} 
                \DecValTok{0} \SpecialCharTok{|} 
\NormalTok{                mt2kids }\SpecialCharTok{\textasciitilde{}}\NormalTok{ samesex,}
              \AttributeTok{data =}\NormalTok{ ae\_pums)}
\FunctionTok{summary}\NormalTok{(mweeks\_tsls)}
\end{Highlighting}
\end{Shaded}

\begin{verbatim}
## TSLS estimation - Dep. Var.: mom_weeks_worked
##                   Endo.    : mt2kids
##                   Instr.   : samesex
## Second stage: Dep. Var.: mom_weeks_worked
## Observations: 402,014
## Standard-errors: IID 
##             Estimate Std. Error  t value              Pr(>|t|)    
## (Intercept) 23.45461   0.579591 40.46754             < 2.2e-16 ***
## fit_mt2kids -6.13733   1.165196 -5.26721 0.0000001385845302174 ***
## whitem      -1.70356   0.216807 -7.85752 0.0000000000000039279 ***
## blackm       5.26127   0.240292 21.89533             < 2.2e-16 ***
## hispm       -3.01084   0.306604 -9.81997             < 2.2e-16 ***
## moreths      2.35626   0.135795 17.35160             < 2.2e-16 ***
## ---
## Signif. codes:  0 '***' 0.001 '**' 0.01 '*' 0.05 '.' 0.1 ' ' 1
## RMSE: 21.9   Adj. R2: 0.030178
## F-test (1st stage), mt2kids: stat = 1,503.8     , p < 2.2e-16 , on 1 and 402,008 DoF.
##                  Wu-Hausman: stat =     2.401e-4, p = 0.987636, on 1 and 402,007 DoF.
\end{verbatim}
\end{frame}

\begin{frame}{2SLS Diagnostics}
\phantomsection\label{sls-diagnostics}
One benefit of using the built-in 2SLS models is that they automatically
compute test-statistics such as the F-statistic.

Recall the intuition of the F-statistic: \emph{Do the explanatory
variables in this regression explain a meaningful amount of variation in
the outcome?}

\begin{enumerate}
\tightlist
\item
  Estimate a ``restricted model'' without the instrument \(Z\) and an
  ``unrestricted model'' with the instrument \(Z\)
\item
  Compare the residual sum of squares (RSS)
\item
  If the RSS drops substantially, the F-stat is large.
\end{enumerate}

If the F-stat is large, this implies that it explains a lot of the
variation in \(Y\).

In other words, it is not a \textbf{weak instrument}

A ballpark F-stat of 10 is usually considered the rule-of-thumb
\end{frame}

\begin{frame}[fragile]{2SLS Diagnostics: Problems with the F-stat}
\phantomsection\label{sls-diagnostics-problems-with-the-f-stat}
However, the F-stat rule-of-thumb is really only valid with certain
assumptions.

In the presence of clustered, heteroskedastic, or a number of other
error cases we likely need a much larger F-stat than 10.

The \texttt{ivDiag} package has some useful methods for more thorough
2SLS diagnostics
\end{frame}

\begin{frame}[fragile]{Advanced 2SLS Diagnostics}
\phantomsection\label{advanced-2sls-diagnostics}
Let's load in the package (don't forget to install if the first time
using it)

\scriptsize

\begin{Shaded}
\begin{Highlighting}[]
\CommentTok{\#install.packages("ivDiag")}
\FunctionTok{library}\NormalTok{(ivDiag)}
\end{Highlighting}
\end{Shaded}

\begin{verbatim}
## ## Tutorial: https://yiqingxu.org/packages/ivDiag/
\end{verbatim}
\end{frame}

\begin{frame}[fragile]{Advanced 2SLS Diagnostics}
\phantomsection\label{advanced-2sls-diagnostics-1}
The \texttt{ivDiag()} function requires us to supply its arguments as
strings.

Let's first use it to look at some F-stats.

\scriptsize

\begin{Shaded}
\begin{Highlighting}[]
\NormalTok{mworked\_ivDiag }\OtherTok{\textless{}{-}} \FunctionTok{ivDiag}\NormalTok{(}\AttributeTok{data =}\NormalTok{ ae\_pums,}
                         \AttributeTok{Y =} \StringTok{"mom\_worked"}\NormalTok{,}
                         \AttributeTok{D =} \StringTok{"mt2kids"}\NormalTok{,}
                         \AttributeTok{Z =} \StringTok{"samesex"}\NormalTok{,}
                         \AttributeTok{controls =} \FunctionTok{c}\NormalTok{(}\StringTok{"whitem"}\NormalTok{, }\StringTok{"blackm"}\NormalTok{, }\StringTok{"hispm"}\NormalTok{, }\StringTok{"moreths"}\NormalTok{),}
                         \AttributeTok{bootstrap =} \ConstantTok{FALSE}\NormalTok{,}
                         \AttributeTok{run.AR =} \ConstantTok{FALSE}\NormalTok{)}
\end{Highlighting}
\end{Shaded}
\end{frame}

\begin{frame}[fragile]{Advanced 2SLS Diagnostics}
\phantomsection\label{advanced-2sls-diagnostics-2}
In addition to returning the OLS, 2SLS, first stage, and reduced form
estimates, the \texttt{ivDiag()} function will return a range of F-stat
tests.

\scriptsize

\begin{Shaded}
\begin{Highlighting}[]
\NormalTok{mworked\_ivDiag}\SpecialCharTok{$}\NormalTok{est\_ols}
\end{Highlighting}
\end{Shaded}

\begin{verbatim}
##             Coef     SE        t CI 2.5% CI 97.5% p.value
## Analytic -0.1215 0.0016 -76.1443 -0.1246  -0.1184       0
\end{verbatim}

\begin{Shaded}
\begin{Highlighting}[]
\NormalTok{mworked\_ivDiag}\SpecialCharTok{$}\NormalTok{est\_2sls}
\end{Highlighting}
\end{Shaded}

\begin{verbatim}
##             Coef    SE       t CI 2.5% CI 97.5% p.value
## Analytic -0.1286 0.026 -4.9445 -0.1795  -0.0776       0
\end{verbatim}

\begin{Shaded}
\begin{Highlighting}[]
\NormalTok{mworked\_ivDiag}\SpecialCharTok{$}\NormalTok{F\_stat}
\end{Highlighting}
\end{Shaded}

\begin{verbatim}
##  F.standard    F.robust   F.cluster F.effective 
##    1503.843    1504.630          NA    1504.630
\end{verbatim}
\end{frame}

\begin{frame}[fragile]{The Anderson-Rubin test}
\phantomsection\label{the-anderson-rubin-test}
If we are estimating an IV with a weak instrument, the standard error on
the 2SLS estimate will not be valid.

Anderson-Rubin (1949) suggest a test statistic which is robust the weak
instrument concern.

The Anderson-Rubin test statistic works backwards:

For a set of possible \(\beta_{0}\) values, it computes if
\(Y - \beta_{0}X\) regressed on \(Z\) is statistically significant.
Then, the set of \(\beta_{0}\) such that \(Z\) is not significant is the
AR Confidence interval.

We can implement this by setting the \texttt{run.AR} argument in
\texttt{ivDiag()} to \texttt{TRUE}
\end{frame}

\begin{frame}[fragile]{The Anderson-Rubin test}
\phantomsection\label{the-anderson-rubin-test-1}
\scriptsize

\begin{Shaded}
\begin{Highlighting}[]
\NormalTok{mworked\_ivDiag\_v2 }\OtherTok{\textless{}{-}} \FunctionTok{ivDiag}\NormalTok{(}\AttributeTok{data =}\NormalTok{ ae\_pums,}
                         \AttributeTok{Y =} \StringTok{"mom\_worked"}\NormalTok{,}
                         \AttributeTok{D =} \StringTok{"mt2kids"}\NormalTok{,}
                         \AttributeTok{Z =} \StringTok{"samesex"}\NormalTok{,}
                         \AttributeTok{controls =} \FunctionTok{c}\NormalTok{(}\StringTok{"whitem"}\NormalTok{, }\StringTok{"blackm"}\NormalTok{, }\StringTok{"hispm"}\NormalTok{, }\StringTok{"moreths"}\NormalTok{),}
                         \AttributeTok{bootstrap =} \ConstantTok{FALSE}\NormalTok{,}
                         \AttributeTok{run.AR =} \ConstantTok{TRUE}\NormalTok{)}
\end{Highlighting}
\end{Shaded}

\begin{verbatim}
## AR Test Inversion...
\end{verbatim}

\begin{verbatim}
## Parallelising on 11 cores
\end{verbatim}

\begin{Shaded}
\begin{Highlighting}[]
\NormalTok{mworked\_ivDiag\_v2}\SpecialCharTok{$}\NormalTok{AR}\SpecialCharTok{$}\NormalTok{ci.print}
\end{Highlighting}
\end{Shaded}

\begin{verbatim}
## [1] "[-0.1795, -0.0776]"
\end{verbatim}
\end{frame}

\end{document}
